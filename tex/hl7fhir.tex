  \section[Fast Healthcare Interoperability Resources (HL7 FHIR\textsuperscript{\texttrademark})]
  {HL7 FHIR\textsuperscript{\texttrademark}}
    \label{sec:fhir}

  \initial{F}\textit{ast Healthcare Interoperability Resources (HL7/FHIR )}\
  is a is a next generation standards framework created by HL7.\
  It defines a set of ``resources'' for health. These resources represent granular clinical\
  concepts that can be exchanged in order to quickly and effectively solve problems in healthcare\
  and related process. The resources cover the basic elements of healthcare - patients, admissions,\
  diagnostic reports, medications, and problem lists, with their typical participants, and also support\
  a range of richer and more complex clinical models. The simple direct definitions of the resources are\
  based on thorough requirements gathering, formal analysis and extensive cross-mapping to other relevant standards. \citep{_fhir_intro_2013}\\

  \noindent Resources are:\

  \begin{itemize}
    \itemsep0ex
    \item Atomic - they are the smallest defined unit of operation and a transaction\
    scope of their own.\
    \item Connected - resources refer to other resources to allow for clean modular\
    reuse of information.\
    \item Independent - the content of a resource can be processed without having to\
    retrieve referenced resources.\
    \item Simple - each resource definition is easy to understand, and to implement\
    without needing specialized tooling or infrastructure (though that can be used if desired).\
    \item RESTful - resources are able to be used in a RESTful exchange context.\
    \item Flexible - resources can also be used in non-RESTful contexts, such as messaging\
    or SOA architectures and can be moved in and out of RESTful paradigms as convenient.\
    \item Extensible - resources can be extended to allow for local requirements without\
    impacting on interoperability.\
    \item Webcentric - where possible and appropriate, open internet standards are used\
    for data representation.\
    \item Free for use - the FHIR specification itself is open - anyone can implement\
    FHIR or derive related specifications without any IP restrictions.\
  \end{itemize}

  \noindent In addition to the basic resources, FHIR defines a lightweight implementation framework that\
  supports the use of these resources in RESTful environments, classic message exchanges, human-centric\
  clinical documents and enterprise SOA architectures. Each of these approaches provides its own benefits\ 
  - FHIR provides the underpinning enablement that makes the choosing one of these painless and enables enterprises\
  to choose their own paradigm without forsaking interoperability with other approaches.\\

  \noindent Though the resources are simple and easy to understand, they are backed by a thorough, global requirements gathering\
  and formal modeling process that ensures that the content of the resources is stable and reliable. The resource\
  contents are mapped to solid underlying ontologies and models using computable languages (including RDF) so\
  that the definitions and contents of the resources can be leveraged by computational analysis and conversion processes.\\

  \noindent FHIR also provides an underlying conformance framework and tooling that allows different implementation contexts\ 
  and enterprises to describe their context and use of resources in formal computable ways and to empower computed\
  interoperability that leverages both the conformance and definitional frameworks.\\

  \noindent The combination of the resources and the 3 supporting layers (implementation frameworks,\
  definitional thoroughness, and conformance tooling), along with the completely free license of FHIR\
  itself frees healthcare data so that it can easily flow to where it needs to be (across hospital production systems,\
  mobile clinical systems, cloud based data stores, national health repositories, research databases, etc.)\
  without having to pass through format and semantic inter-conversion hurdles along the way. \citep{_fhir_framework_2013}\
