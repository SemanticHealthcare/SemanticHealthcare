%----------------------------------------------------------------------------------------
%	PACKAGES AND OTHER DOCUMENT CONFIGURATIONS
%----------------------------------------------------------------------------------------
\documentclass[DIV=calc, paper=a4, fontsize=12pt, onecolumn]{scrartcl}	 % A4 paper and 12pt font size

\usepackage[english]{babel} % Engllish language /hyphenation
\usepackage[protrusion=true,expansion=true]{microtype} % Better typography
\usepackage{booktabs} % Horizontal rules in tables
\usepackage[svgnames]{xcolor} % Enabling colors by their 'svgnames'
\usepackage{fix-cm}	 % Custom font sizes - used for the initial letter in the document

\usepackage{sectsty} % Enables custom section titles
\allsectionsfont{\usefont{OT1}{phv}{b}{n}} % Change the font of all section commands

\usepackage{fancyhdr} % Needed to define custom headers/footers
\pagestyle{fancy} % Enables the custom headers/footers
\usepackage{lastpage} % Used to determine the number of pages in the document (for "Page X of Total")
\usepackage{url} % Used to format urls correctly

% Headers - all currently empty
\lhead{}
\chead{}
\rhead{}

% Footers
\lfoot{}
\cfoot{}
\rfoot{\footnotesize Page \thepage\ of \pageref{LastPage}} % "Page 1 of 2"

\renewcommand{\headrulewidth}{0.0pt} % No header rule
\renewcommand{\footrulewidth}{0.4pt} % Thin footer rule

\usepackage{hyperref}
\hypersetup{
	colorlinks=true, %set true if you want colored links
	linktoc=all, %set to all if you want both sections and subsections linked
	linkcolor=blue, %choose some color if you want links to stand out
	citecolor=Maroon, 
	}
	
\addto{\captionsenglish}{\renewcommand*\contentsname{Table of Contents}}

\usepackage{lettrine} % Package to accentuate the first letter of the text
\newcommand{\initial}[1]{ % Defines the command and style for the first letter
\lettrine[lines=3,lhang=0.3,nindent=0em,slope=0em]{
\color{DarkBlue}
{\textbf{\textit{#1}}}}{}}


%----------------------------------------------------------------------------------------
%	TITLE SECTION
%----------------------------------------------------------------------------------------

\usepackage{titling} % Allows custom title configuration

\newcommand{\HorRule}{\color{DarkGoldenrod} \rule{\linewidth}{1pt}} % Defines the gold horizontal rule around the title

\pretitle{ \vspace{-150pt}\begin{flushleft} \HorRule \fontsize{40}{40} \usefont{OT1}{phv}{b}{n} \color{DarkRed} \selectfont} % Horizontal rule before the title

\title{Semantic Healthcare} % Your article title

\posttitle{\par\end{flushleft}\vskip 0.5em} % Whitespace under the title

\preauthor{\begin{flushleft}\large \lineskip 0.5em \usefont{OT1}{phv}{b}{sl} \color{DarkRed}} % Author font configuration

\newcommand{\org}{\footnotesize \usefont{OT1}{phv}{m}{sl} \color{Black}}

\DeclareRobustCommand{\authoring}{
\begin{tabular}{ccc}
  Sankhesh Jhaveri & Catherine Dumas & Joshua Cope\\
  \org Kitware,Inc. & \org SUNY Albany & \org SUNY Albany
\end{tabular}}

\author{\authoring} % Your name

\postauthor{\footnotesize \usefont{OT1}{phv}{m}{sl} \color{Black} % Configuration for the institution name
 % Your institution

\par\end{flushleft}\HorRule} % Horizontal rule after the title

\date{} % Add a date here if you would like one to appear underneath the title block

\setcounter{secnumdepth}{0} % All sections start from depth 1

%----------------------------------------------------------------------------------------

\begin{document}


\maketitle

\begin{minipage}[t]{\textwidth}
\center{\bfseries\contentsname}
\begin{center}
\makeatletter
\@starttoc{toc}
\makeatother
\end{center}
\end{minipage}

\thispagestyle{fancy} % Enabling the custom headers/footers for the first page
%----------------------------------------------------------------------------------------
\section[Systematized Nomenclature in Medicine - Clinical Terms ( SNOMED CT )]{SNOMED CT}
\section[Systematized Nomenclature in Medicine - Reference Terminology ( SNOMED  RT )]{SNOMED RT}
\section[Healthcare Data Dictionary ( HDD )]{HDD}
\section[Convergent Medical Terminology ( CMT )]{CMT}
\section[International Classification of Diseases ( ICD )]{ICD}
\section[Logical Observation Identifier Names and Codes terminology ( LOINC )]{LOINC}
\section[Bio2RDF]{BIo2RDF}
\section[Fast Health Interoperable Resources (HL7/FHIR)]{HL7/FHIR}
\section[Nationwide Health Information Network ( NwHIN )]{NwHIN}
\section[Resource Description Framework (RDF)]{RDF}
\section[SPARQL Protocol and RDF Query Language (SPARQL)]{SPARQL}
\section[FileMan Query Language ( FMQL )]{FMQL}

\initial{S}\textit{ystematized Nomenclature in Medicine - Clinical Terms ( SNOMED CT )} is a comprehensive, multilingual clinical terminology that provides clinical content and expressivity for clinical documentation and reporting. It can be used to code, retrieve and analyze clinical data. In a nutshell, SNOMED CT consists of concepts arranged in a hierarchy, connected by relationships. The International Health Terminology Standards Development Organization (IHTSDO) owns and administers the rights to SNOMED CT.\\

There are three basic components of SNOMED CT:
\begin{itemize}
	\item{Concepts}
	\item{Descriptions}
	\item{Relationships}
\end{itemize}

\textbf{Concepts} are clinical ideas, ranging from abscess to zygote, identified by a unique numeric identifier (\textit{ConceptId}) that never changes and represented by a unique human readable \textit{Fully Specified Name (FSN)}\cite{snomed_-_user_guide_snomed_2011,snomed_implementation_guide_snomed_2011}.


%----------------------------------------------------------------------------------------
%	REFERENCE LIST
%----------------------------------------------------------------------------------------

\bibliographystyle{apalike}
\bibliography{SemanticHealthcare}{}

%----------------------------------------------------------------------------------------

\end{document}
